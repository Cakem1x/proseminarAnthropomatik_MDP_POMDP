%% LaTeX-Beamer template for KIT design
%% by Erik Burger, Christian Hammer
%% title picture by Klaus Krogmann
%%
%% version 2.0
%%
%% mostly compatible to KIT corporate design v2.0
%% http://intranet.kit.edu/gestaltungsrichtlinien.php
%%
%% Problems, bugs and comments to
%% burger@kit.edu

\documentclass[18pt]{beamer}
\usetheme{kit}

\titleimage{titleImage}
\titlelogo{logos}

% the presentation starts here

\title[MDP Pfadplanung]{Markov-Entscheidungsprozesse für die Roboterpfadplanung}
\subtitle{Entscheidungsprobleme unter Unsicherheit} %TODO nötig?
\author{Matthias Holoch}

\institute{Proseminar Anthropomatik: Von der Theorie zur Anwendung}

\begin{document}

\selectlanguage{ngerman}

%title page
\begin{frame}
	\titlepage
\end{frame}

%table of contents %TODO remove it, maybe
\frame{
	\frametitle{Gliederung}
	\tableofcontents
}


\section{Einleitung}
\subsection{Wozu das Ganze?}
\begin{frame}
	\frametitle{Wozu das Ganze?}
	\begin{itemize}
		\item Lösen von Entscheidungsproblemen 
		\begin{itemize}
			\item Einbezug von Unsicherheiten
		\end{itemize}
		\visible<2->{
		\item Medizinische Diagnose
		\begin{itemize}
			\item genauer Patientenzustand unbekannt
			\item Menge an Therapiemöglichkeiten
			\item Beobachtungen als Hinweis auf den Patientenzustand
			\item Abwägen von Kosten
		\end{itemize}
		}
		\visible<3->{
		\item Roboterpfadplanung
		}
	\end{itemize}
\end{frame}


\section{Planung unter Unsicherheit}
\subsection{Planung klassisch}
\begin{frame}
	\frametitle{Klassische Pfadplanung}	
	\visible<2->{
		\only<1,2>{
			\includegraphics[scale=0.8]{images/autnmRobot_basicSituation.png}
		}
		\only<3->{
			\includegraphics[scale=0.8]{images/autnmRobot_directPath.png}
		}
	}
	\only<-4>{
		\\Annahmen
		\begin{enumerate}
			\item Genauer Zustand erkennbar
			\item Aktionsausführung fehlerfrei
		\end{enumerate}
		\visible<4->{
			Probleme
			\begin{itemize}
				\item Fehler im Bewegungsmodell
				\item Schlupf zwischen Antriebsrädern und Untergrund
				\item verrauschte Sensordaten
			\end{itemize}
		}
	}
	\only<5->{
		\\ Aktionsausführung nicht fehlerfrei, genauer Zustand nicht erkennbar.
		\begin{itemize}
			\item Verlassen des geplanten Pfads
			\item häufige Korrektur
			\item erhöhtes Messintervall bei engen Korridoren
		\end{itemize}
		\visible<6>{
			\textbf{Einzelne Aktionsfolge reicht nicht aus.}
		}
	}
\end{frame}

\subsection{Pfadplanung mit MDPs}
\begin{frame}
	\frametitle{Pfadplanung mit MDPs}
	\visible<3->{
		\only<1,2,3>{
			\includegraphics[scale=0.8]{images/autnmRobot_basicSituation.png}
		}
		\only<4>{
			\includegraphics[scale=0.8]{images/autnmRobot_detActionMDP.png}
		}
		\only<5>{
			\includegraphics[scale=0.8]{images/autnmRobot_ndetActionMDP.png}
		}
	}
	\\Annahmen
	\begin{enumerate}
		\item Genauer Zustand erkennbar
	\end{enumerate}
	\visible<2->{
		Vorberechnete Aktionsfolgen
		\begin{itemize}
			\item antizipieren Fehlverhalten
			\item sind \glqq optimal\grqq
		\end{itemize}
	}
\end{frame}


\section{MDP}
\subsection{Formal}
\begin{frame}
	\frametitle{MDP}
	Ein MDP ist ein 4-Tupel
	\begin{equation}
		(S, A, T, R)
	\end{equation}
	mit
	\begin{itemize}
		\visible<1->{
			\item diskreter, endlicher Zustandsmenge $S$
		}
		\visible<2->{
			\item diskreter, endlicher Aktionsmenge $A$
		}		
		\visible<3->{
			\item Zustandsübergangsmodell des Systems $T$
				\begin{equation}
					T(s, a, s') = \mathbf{p}(s'|s, a)
				\end{equation}
		}
		\visible<4->{
			\item Funktion
				\begin{equation}
					r: S \times A \rightarrow \mathbb{R}
				\end{equation}
		}
	\end{itemize}
\end{frame}

\subsection{Beispiel}
\begin{frame}
	\frametitle{MDP Beispiel}
	\only<1>{
		\includegraphics[scale=0.2]{images/MDP_example_part2.png}
	}
	\only<2>{
		\includegraphics[scale=0.2]{images/MDP_example_part1.png}
	}
	\only<3>{
		\includegraphics[scale=0.2]{images/MDP_example_part0.png}
	}
	\only<4>{
		\includegraphics[scale=0.2]{images/MDP_example.png}
	}
	\begin{itemize}
	\visible<1->{
		\item $S = \{s_0, s_1, s_2\}$
	}
	\visible<2->{
		\item $A = \{a_0, a_1\}$
	}
	\visible<3->{
		\item $T(s_0, a_0, s_0) = 0.2;\ T(s_0, a_0, s_1) = 0.8;\ ...$
	}
	\visible<4->{
		\item $r(s_0, a_1) = 1;\ r(s_2, a_0) = 10;\ r(s_1, a_1) = -5$
	}
	\end{itemize}
\end{frame}

\section{Strategien}
\subsection{Optimale Strategie}

\subsection{Algorithmus}


\section{Schluss}
\subsection{Ausblick}

\subsection{Quellen}
\begin{frame}{~}
		\bibliographystyle{plain}
		\bibliography{literatur}
\end{frame}

\end{document}
