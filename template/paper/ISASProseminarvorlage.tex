\documentclass[a4paper]{IEEEtran}


% Ein paar hilfreiche Pakete
\usepackage{german}
\usepackage[latin1]{inputenc}
\usepackage{graphicx} 
\usepackage{amsmath} 
\usepackage{amssymb}  
\usepackage{mathtools}
\mathtoolsset{showonlyrefs}
\usepackage{subfigure}
\usepackage{flushend}
\usepackage{url}

% Ein paar am ISAS �bliche Formelzeichen
\def\rv#1{{\mathbf #1}} %Random Variable
\def\vec#1{\underline{#1}} %Vector
\def\rvv#1{{\vec{\rv{#1}}}} %Random Vector
\def\mat#1{{\mathbf #1}} %Matrix
\def\Var{\mathrm{Var}} %Variance
\def\E{\mathrm{E}} %Expectation
\def\Cov{\mathrm{Cov}} %Covariance
\def\IN{\mathrm{I\hspace{-2pt}N}} %Natural Numbers
\def\IR{\mathrm{I\hspace{-2pt}R}} %Real Numbers 

% correct bad hyphenation here
\hyphenation{op-tical net-works semi-conduc-tor}


\begin{document}
\title{Thema des Beitrags}

\author{Max~Musterstudent,~\IEEEmembership{E-Mail: max.musterstudent@ira.uka.de}}% <-this % stops a space




% The paper headers
\markboth{Proseminar WS 12/13: Anthropomatik: Von der Theorie zur Anwendung}%
{Proseminar WS 12/13: Anthropomatik: Von der Theorie zur Anwendung}



% make the title area
\maketitle


\begin{abstract}
Die Ausarbeitung beginnt mit einer kurzen Zusammenfassung.
\end{abstract}







\section{Einleitung}
Hier beginnt der Text...

\section{Ein paar Hinweise}

\subsection{Abs�tze, etc.}
Ein neuer Absatz sollte nicht durch einen Zeilenumbruchs-Befehl sondern durch eine Leerzeile im Code erzeugt werden.
Vor und nach Formeln sollte durch Kommentarmarken "`\%"' sichergestellt werden, dass kein neuer Absatz beginnt (es sei denn, dies ist explizit gew�nscht).
Also
%
\begin{equation}
  a = b + 3~,
\end{equation}
%
und nicht 

\begin{equation}
  a = b + 4~,
\end{equation}

was komisch aussehen w�rde. 


\subsection{Mathematischer Formelsatz}
Bitte bei der Ausarbeitung die vorgefertigten Makros verwenden:

Vektoren und Matrizen
%
\begin{equation}
  \vec{x}, \mat{A}
\end{equation}
%
Mengenzeichen
%
\begin{equation}
  \IR, \IN
\end{equation}
%
Zufallsvariablen, etc...
%
\begin{equation}
  \rv{y}, \rvv{z},
  \Var, \E, \Cov
\end{equation}
%
Bitte nur Gleichungen nummerieren, auf die sich auch sp�ter bezogen wird (sollte automatisch geschehen)
%
\begin{equation}
  \label{eq:NameDergleichung}
  a = b + c ~.
\end{equation}
%
Laut \eqref{eq:NameDergleichung} ist $a=b+c$.

Mehrzeiliger Formelsatz mit \emph{align}
%
\begin{align}
  a &= b + c \enspace ,\\
  a_{ij} &= b_{ij} + c_{ij} \enspace .
\end{align}
%
oder mit \emph{multline}
%
\begin{multline}
  a_{2343443} = \\
  b + c + \frac{346}{324557} 
  \cdot \left( b_{ij} + c_{ij} \right)\\
  \cdot \int_{x=55}^{88} x^{67823+x} \frac{x}{32455767567567575677} \text{d}x
  \enspace .
\end{multline}
%

Funktionen sollen in Formeln \emph{nicht} kursiv gesetzt werden. Dazu gibt es in LaTeX f�r fast alle Funktionen schon Makros, z.B. 
%
\begin{align}
  y= \sin (x)
\end{align}
%
und nicht 
\begin{align}
  y= sin (x)~.
\end{align}

%So werden Bilder eingebunden (als pdf, jpg oder png)
%\begin{figure}[ht]
%  \centering
%  \caption{Hier kommen weitere Erkl�rungen zum Bild}
% \label{fig:autorname_bild1}
%\end{figure}
%
% Auf diese Abbildung wird dann mit Abb. \ref{fig:autorname_bild1} verwiesen.

\section{Zitate}
Bitte immer korrekt zitiernen \cite{ITM07_BrunnSawo,IPSN07_Huber}!

\section{Zusammenfassung und Ausblick}


%%%%%%%%%%%%%%%%%%%%%%%%%%%%%%%%%%%%%%%%%%%%%%%%%%%%%%%%%%%%%%%%%%%%%%%%%
% Literaturverzeichnis (in literatur.bib, z.B. mit Jabref editieren) 
\bibliographystyle{plain}
\bibliography{literatur}
\end{document}


